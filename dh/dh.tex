\documentclass[a4paper]{report}
\usepackage{amsmath}
\usepackage{amsfonts}
\usepackage{ngerman}
\usepackage{tikz}

\begin{document}
\chapter*{Diffie-Hellman-Schlüsselaustausch}
\section*{Allgemein}
\begin{center}
\begin{tabular}{c|c|c}
    öffentlich & Alice & Bob \\
    \hline
    $g\in\mathbb{P},\quad p\in\mathbb{N},\quad p<g$ &
    $x\in\mathbb{N}$ &
    $y\in\mathbb{N}$ \\ & & \\
    &
    $a=g^x \mod p$ &
    $b=g^y \mod p$ \\
    & $s_1=b^x \mod p$ & $s_2=a^y \mod p$
\end{tabular}
\end{center}

\section*{Beispiel}
\begin{center}
\begin{tabular}{c|c|c}
    öffentlich & Alice & Bob \\
    \hline
    $g=13,\quad p=9$ &
    $x=4$ &
    $y=3$ \\ & & \\
    &
    $a=13^4 \mod 9=4$ &
    $b=13^3 \mod 9=1$ \\
    & $s_1=1^4 \mod 9=1$ & $s_2=4^3 \mod 9=1$
\end{tabular}
\end{center}

\section*{Wieso ist $s_1=s_2$?}
\begin{align}
    s_1&=\left(g^y \mod p\right)^x \mod p \\
    s_1&=\left(g^y\right)^x \mod p \\
    s_1&=g^{yx} \mod p \\ \\
    s_2&=\left(g^x \mod p\right)^y \mod p \\
    s_2&=\left(g^x\right)^y \mod p \\
    s_2&=g^{xy} \mod p \\ \\
    s_1&=s_2
\end{align}
\end{document}
