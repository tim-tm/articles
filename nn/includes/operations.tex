\chapter{Bitweise Operationen}
Eine bitweise Operation kann auf einen Bit oder eine Kette von Bits angewandt werden.
Je nach Operation können durch bitweise Operationen, Bitketten beispielsweise umgekehrt werden.

In den folgenden Abbildungen steht $x$ bzw. $x_1$, $x_2$ für die Bits auf die, die jeweilige Operation angewandt werden soll und $y$ für das Ergebnis.

\begin{figure}[ht]
    \centering
    \begin{tabular}{|c|c|}
        \hline
        x & y \\
        \hline
        0 & 1 \\
        \hline
        1 & 0 \\
        \hline
    \end{tabular}
    \caption{NICHT-Operation}
\end{figure}

\begin{figure}[ht]
    \centering
    \begin{tabular}{|c|c|c|}
        \hline
        $x_1$ & $x_2$ & y \\
        \hline
        0 & 0 & 0 \\
        \hline
        0 & 1 & 0 \\
        \hline
        1 & 0 & 0 \\
        \hline
        1 & 1 & 1 \\
        \hline
    \end{tabular}
    \caption{UND-Operation}
\end{figure}

\begin{figure}[ht]
    \centering
    \begin{tabular}{|c|c|c|}
        \hline
        $x_1$ & $x_2$ & y \\
        \hline
        0 & 0 & 0 \\
        \hline
        0 & 1 & 1 \\
        \hline
        1 & 0 & 1 \\
        \hline
        1 & 1 & 1 \\
        \hline
    \end{tabular}
    \caption{ODER-Operation}
    \label{fig:or}
\end{figure}

\begin{figure}[ht]
    \centering
    \begin{tabular}{|c|c|c|}
        \hline
        $x_1$ & $x_2$ & y \\
        \hline
        0 & 0 & 0 \\
        \hline
        0 & 1 & 1 \\
        \hline
        1 & 0 & 1 \\
        \hline
        1 & 1 & 0 \\
        \hline
    \end{tabular}
    \caption{XOR-Operation}
\end{figure}

